\documentclass[11pt,a4paper]{article}
\usepackage[utf8]{inputenc}
\usepackage{amsmath}
\usepackage{amsfonts}
\usepackage{amssymb}
\author{Santiago Sanz Wuhl}
\title{Práctica 4: Calor de Reacción por Calorimetría}
\begin{document}
\maketitle
\section{Objetivo}
\quad El objetivo de esta práctica es medir cuál es el calor desprendido en la reaccioń de $H_{2}SO_{4}$ con $NaOH$. Para ello tendremos primero que conocer cuál es el equivalente en agua del calorímetro que usaremos. El equivalente en agua de un calorímetro se define como la masa de agua para la cual, con el mismo calor, aumenta la misma cantidad de grados. Sabiendo esto, mezclaremos en el calorímetro agua fría con agua caliente, lo dejaremos reposar y mediremos la temperatura final. 

Si tuviesemos un sistema perfectamente aislado, la temperatura final correspondería con la temperatura media entre las dos masas de agua(por que mezclaremos la misma cantidad de agua fría que de agua caliente), pero como no disponemos de uno, medimos que la temperatura de equilibrio será menor que la media, por que parte del calor cedido por el agua caliente será absorbido por el calorímetro. Usando esto mediremos el equivalente en masa de agua del calorímetro. 
\section{Fundamento Teórico}
\quad El calor desprendido por el agua caĺiente viene dado por la siguiente expresión:
\begin{center}
$Q_{c} = Q_{f} + Q_{t}$ \\
$m_{c}C_{e}(t_{e}-t_{c})=m_{f}C_{e}(t_{e}-t_{f})+m_{E}C_{e}(t_{e}-t{f})$ \\
\end{center}
De donde deducimos que el equivalente en agua del calorímetro será:
\begin{center}
$m_{E} = \frac{m_{c}t_{e}-t_{c}}{t_{e}-t_{f}} - m_{f}$
\end{center}
\section{Desarrollo de la Práctica}
Vertemos en el calorímetro 200 mL de agua. Esto nos servirá para comprobar que podemos colocar el termómetro a una altura cómoda, de forma que no nos entorpezca al medir la temperatura de equilibrio. Una vez establecida una altura cómoda, pasamos el termómetro por agujero del tapón del calorímetro y lo fijamos ahí con 
\quad 
\end{document}